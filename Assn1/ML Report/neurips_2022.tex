\documentclass{article}



% if you need to pass options to natbib, use, e.g.:
%     \PassOptionsToPackage{numbers, compress}{natbib}
% before loading neurips_2022


% ready for submission
\usepackage[preprint]{neurips_2022}


% to compile a preprint version, e.g., for submission to arXiv, add add the
% [preprint] option:
%     \usepackage[preprint]{neurips_2022}


% to compile a camera-ready version, add the [final] option, e.g.:
%     \usepackage[final]{neurips_2022}


% to avoid loading the natbib package, add option nonatbib:
%    \usepackage[nonatbib]{neurips_2022}
\usepackage{mathptmx}
\usepackage{graphicx}
\usepackage{anyfontsize}
\usepackage{t1enc}
\usepackage{amsthm}
\usepackage{amssymb}
\usepackage[utf8]{inputenc} % allow utf-8 input
\usepackage[T1]{fontenc}    % use 8-bit T1 fonts
\usepackage{hyperref}       % hyperlinks
\usepackage{url}            % simple URL typesetting
\usepackage{booktabs}       % professional-quality tables
\usepackage{amsfonts}       % blackboard math symbols
\usepackage{nicefrac}       % compact symbols for 1/2, etc.
\usepackage{microtype}      % microtypography
\usepackage{xcolor}         % colors
\usepackage{amsmath}
\usepackage{hyperref}      % To remove red rectangle around contents
\hypersetup{pdfborder=0 0 0}
\graphicspath{ {./images/} }


\title{Assignment 1}


% The \author macro works with any number of authors. There are two commands
% used to separate the names and addresses of multiple authors: \And and \AND.
%
% Using \And between authors leaves it to LaTeX to determine where to break the
% lines. Using \AND forces a line break at that point. So, if LaTeX puts 3 of 4
% authors names on the first line, and the last on the second line, try using
% \AND instead of \And before the third author name.


\author{%
 % David S.~Hippocampus\thanks{Use footnote for providing further information
   % about author (webpage, alternative address)---\emph{not} for acknowledging
   % funding agencies.} \\
  Indian Institute of Information, Kanpur\\
  Department of Computer Science\\
  CS771 Introduction to Machine Learning\\
  Instructor: Purushottam Kar \\\\ 
  Submitted: September 17, 2022
  %Pittsburgh, PA 15213 \\
  %\texttt{hippo@cs.cranberry-lemon.edu} \\
  % examples of more authors
  \AND
  Ambati Madhav  \\
 160098\\
  % \texttt{email} \\
  \And
  Kudupudi Mohan Kumar \\
  % Affiliation \\
  22105037 \\
  % \texttt{email} \\
   \And
  Abhinav Kuruma \\
  % Affiliation \\
  22111401 \\
 \AND
  % \texttt{email} \\
  Gnanendra Sri Phani Sai Channamsetty \\
  % Affiliation \\
  22105032 \\
  \And
  Rahul Aggarwal\\
 22111403\\
}


\begin{document}
\maketitle


%Sections \ref{gen_inst}, \ref{headings}, and \ref{others} below.
\newpage
\tableofcontents
\newpage
\section{Answer 1}
As we have to map binary digits 0,1 to signs -1,+1,\\
We should make a function say, $m:\{0,1\} \to \{-1,+1\}$. Such that \\
$$
m(x)=\begin{cases}
			-1, & \text{if $x=1$ }\\
            +1, & \text{if $x=0$}\\
	   \end{cases}
$$
and another function named $f:\{-1,+1\} \to \{0,1\}$ and it should be such that \\
$$
f(x)=\begin{cases}
			0, & \text{if $x=+1$ }\\
            1, & \text{if $x=-1$}\\
		 \end{cases}
$$

So, for $m$ the fuction is $m(x) =1 -2x$ and for $f$ the function is $f(x) = \frac{1-sign(x)}{2}$.\\\\
And, we also observe that $f$ is the inverse of $m$. Now, let's take an example to see that
\begin{equation}
XOR(b_1,b_2, \dots ,b_n) = f\left( \prod_{i=1}^n m(b_i) \right)
\end{equation}
\\
Assume $b_1=0,b_2=1,b_3=1$

\begin{equation} \label{eq1}
\begin{split}
XOR(0,1,1) & = f( m(0) * m(1) * m(1) )\\
& = f(1*-1*-1) \hspace{2cm} \text{$\because$} m(x) = 1-2x\\
& = f(1)\\
& = 0  \hspace{4cm} \text {$\because$} f(x) = \frac{1-sign(x)}{2}\\
\end{split}
\end{equation}
\\
Hence, we can implement XOR in this manner.


\section{Answer 2}
\label{gen_inst}
\textbf{(To prove)}  To exploit the above result, first give a mathematical proof that for any real numbers
(that could be positive, negative, zero) \emph{$r_1$, $r_2$, . . . ,$ r_n$} for any n $\in$ N , we always have
\[\prod_{i=1 }^n sign(r_i) = sign \prod_{i=1}^n r_i\]

\paragraph{(Proof)}
To get the sign of a number we can use the following:
\begin{equation} \label{eq2}
sign(r)= \frac{|r|}{r}
\end{equation}
where,
$$
sign(r)=\begin{cases}
			-1, & \text{if $r<0$ }\\
            1, & \text{if $r>0$}\\
	0, & \text{if $r=0$}
		 \end{cases}
$$
Using above definition we can say that,
\[ \frac{|r_1||r_2|...|r_n|}{r_1r_2...r_n} = \frac {|r_1r_2...r_n|}{r_1r_2...r_n}\]
proving this will prove the theorem.\\

\subsection{Case 1}

$ \text{For,} \, r_i \ne 0  \;\, \forall_{i=0}^n r_i $
\begin{equation} \label{eq3}
\begin{split}
\text{ L.H.S } & =  \prod_{i=1 }^n sign(r_1)  \\
& = sign(r_1)*sign(r_2)*\dots *sign(r_n)  \\
& = \frac{|r_1|}{r_1}*\frac{|r_2|}{r_2}*\dots * \frac{|r_n|}{r_n}\\
& = \frac{|r_1||r_2|...|r_n|}{r_1r_2...r_n}\, \hspace{1cm} \text{($\because$ Product term amount to positive value)}\\
\end{split}
\end{equation}
\[sign \, \text{of L.H.S depends on}\, (\frac{1}{r_1r_2...r_n})\]
\\
\begin{equation} \label{eq4}
\begin{split}
\text{ R.H.S } & =  sign \prod_{i=1}^n r_i \\
& = sign[(r_1)*(r_2)*\dots *(r_n)]  \\
& = \frac{|r_1*r_2* \dots *r_n|}{r_1*r_2* \dots *r_n} \hspace{7cm}\\
& = \frac{|r_1r_2\dots r_n|}{r_1r_2...r_n}  \\
\end{split}
\end{equation}

Let $r_1*r_2* \dots *r_n $ = x,\\
Then, we get from equation \ref{eq2}  $\frac{|x|}{x}$, and \\
$$
|x|=\begin{cases}
			-x, & \text{if $x<0$ }\\
            x, & \text{if $x>0$}\\
		 \end{cases}
$$
So, anyways numerator becomes positive. As it is a modular function. So, \\
\[ sign \, \text{of R.H.S depends on} \, (\frac{1}{r_1r_2 \dots r_n}) \hspace{2cm} \text{($\because$ The numerator itself is positive)} \]\\\\
\hspace*{6cm}  L.H.S=R.H.S\\ %mkc

\subsection{Case 2} 

$\text{For,} \,  r_i = 0  \;\, \forall_{i=0}^n r_i $ \\\\ As $sign(0)=0$. So, we don't care about other values and simply the whole value becomes zero for both L.H.S and R.H.S.\\\\
\hspace*{6cm}  L.H.S=R.H.S\\
\hspace*{6cm} Hence Proved.



\section{Answer 3}%\label{headings}
We have to prove that, we can map 9-Dimensional vector to D-Dimensional vector:\\
\[R^9 \to R^D\]
\\
$\text{such that,} \, \forall \, (\tilde{u},\tilde{v},\tilde{w}), \exists \, w \in R^D$\\ %Gaand fat gyi
Mathematically,
\begin{equation}
(\tilde{u}^T\tilde{x}).(\tilde{v}^T\tilde{x}).(\tilde{w}^T\tilde{x}) = w^T.\phi(\tilde{x})\\
\end{equation}

\begin{align*} 
L.H.S & = (\tilde{u}^T\tilde{x}).(\tilde{v}^T\tilde{x}).(\tilde{w}^T\tilde{x})\\
& = (\Sigma_{i=1}^9 \tilde{u}_i\tilde{x}_i)(\Sigma_{j=1}^9 \tilde{u}_j\tilde{x}_j)(\Sigma_{k=1}^9 \tilde{u}_k\tilde{x}_k)\\
& = \Sigma_{i=1}^9\Sigma_{j=1}^9\Sigma_{k=1}^9 \, \tilde{u}_i\tilde{u}_j\tilde{u}_k\tilde{x}_i\tilde{x}_j\tilde{x}_k\\
\end{align*}

We have to map from $\tilde{x} \to \phi(\tilde{x})$ such that $\tilde{x}$ is a 9-Dimensional and $\phi(\tilde{x})$ is 729 dimension.\\
So, we get\\\\
$\tilde{x} = (\tilde{x}_1\tilde{x}_2 \dots \tilde{x}_9) \to\phi(\tilde{x}) = (\tilde{x}_1\tilde{x}_1\tilde{x}_1, \tilde{x}_1\tilde{x}_1\tilde{x}_2, \dots , \tilde{x}_1\tilde{x}_1\tilde{x}_9, \tilde{x}_1\tilde{x}_2\tilde{x}_1, \dots ,\tilde{x}_9\tilde{x}_9\tilde{x}_9) $\\

Now, we get
\[(\tilde{u}^T\tilde{x}).(\tilde{v}^T\tilde{y}).(\tilde{w}^T\tilde{z}) = w^T.\phi(\tilde{x})\]\\
where, w = $(\tilde{u}_1\tilde{v}_1\tilde{w}_1, \tilde{u}_1\tilde{v}_1\tilde{w}_2, \dots , \tilde{u}_1\tilde{v}_1\tilde{w}_9, \tilde{u}_1\tilde{v}_2\tilde{w}_1, \dots\ , \tilde{u}_9\tilde{v}_9\tilde{w}_9)$\\

and so w is of 729 dimension. So, we proved\\
\[(\tilde{u}^T\tilde{x}).(\tilde{v}^T\tilde{y}).(\tilde{w}^T\tilde{z}) = w^T.\phi(\tilde{x})\]\\
i.e we can map 9-Dimensional to 729 Dimensional vector.\\\\
where $ \exists \, w \in R^D \, \text{for any } (\tilde{u},\tilde{v},\tilde{w} ) \text{ such that } \forall \, \tilde{x} \in R^9. $

\section{Answer 4}
<Code Link>
\section{Answer 5}
\label{A5}
\subsection{Hyperparameters used in Code(Q4)}
There are 3 Hyperparameters in the code submitted\\
\begin{itemize}
\item[1)] Step Length or Learning Rate 'lr'
\item[2)] Correction Factor or lambda 'la'
\item[3)] Dynamic Epoch '\_epoch'
\end{itemize}


\subsection{Explanation for selection of Hyperparameters}

\begin{itemize}
\item[1)] To obtain optimised values of Step Length and Correction Factor or Lambda, \textbf{grid search} has been used.
\item Step length 'lr' was iterated in three different ranges of $[0.20e^{-01} \; to \; 0.29e^{-01}],[0.20e^{-02}\;  to \; 0.29e^{-02}],[0.20e^{-03} \; to \; 0.29e^{-03}]$
\item Correction Factor 'la' was iterated in three different ranges of $[0.20e^{-01} \; to \; 0.29e^{-01}],[0.20e^{-02}\;  to \; 0.29e^{-02}],[0.20e^{-03} \; to \; 0.29e^{-03}]$
\item Step length value of $0.23e^{-02}$ and Correction Factor value of $0.20e^{-03}$ have been obtained as optimised values from this grid search operation
\item[2)] To decide the total number of epochs '\_epoch', we chose a dynamic epoch which varies based on number of input data points using ceil() function as 
\[\_epochs = \lceil(\frac{epochs*10000}{y.size})\rceil\]
\item To arrive at ‘epochs’ value of 5, a curve was plotted between hinge loss and number of epochs for 10000 train datapoints, number of epochs varying from 2 to 10. It was found that at epochs 5, model reached maximum of 100 percent accuracy , then decreased in accuracy followed by a subsequent increase in accuracy. It was speculated that model started overfitting beyond epoch 5. Hence,  5 has been selected as ‘epochs’ value.
\item Plot between Hinge Loss and Number of Epochs has been included below\\
\\
\includegraphics[scale=0.4]{images/Loss.jpeg}
\end{itemize}

\subsection{Explanation for initialization of weights and bias term}
Both the Weights and bias term have been initialised with zero as it was found to be the fastest in converging to a optimised solution.

\section{Answer 6}
\label{A6}
\subsection{Convergence Curve}
Convergence Curve between Time taken and test classificaton Accuracy has been included below.\\
GD stands for Gradient Descent solver method\\
\\
\includegraphics[scale=1]{images/convergence.png}







\section{Citations, figures, tables, references}
\subsection{Figures}

\begin{figure}
  \centering
  \fbox{\rule[-.5cm]{0cm}{4cm} \rule[-.5cm]{4cm}{0cm}}
  \caption{Sample figure caption.}
\end{figure}

\begin{ack}
Use unnumbered first level headings for the acknowledgments. All acknowledgments
go at the end of the paper before the list of references. Moreover, you are required to declare
funding (financial activities supporting the submitted work) and competing interests (related financial activities outside the submitted work).
More information about this disclosure can be found at: \url{https://neurips.cc/Conferences/2022/PaperInformation/FundingDisclosure}.


Do {\bf not} include this section in the anonymized submission, only in the final paper. You can use the \texttt{ack} environment provided in the style file to autmoatically hide this section in the anonymized submission.
\end{ack}


\section*{References}


References follow the acknowledgments. Use unnumbered first-level heading for
the references. Any choice of citation style is acceptable as long as you are
consistent. It is permissible to reduce the font size to \verb+small+ (9 point)
when listing the references.
Note that the Reference section does not count towards the page limit.
\medskip


{
\small


[1] Alexander, J.A.\ \& Mozer, M.C.\ (1995) Template-based algorithms for
connectionist rule extraction. In G.\ Tesauro, D.S.\ Touretzky and T.K.\ Leen
(eds.), {\it Advances in Neural Information Processing Systems 7},
pp.\ 609--616. Cambridge, MA: MIT Press.


[2] Bower, J.M.\ \& Beeman, D.\ (1995) {\it The Book of GENESIS: Exploring
  Realistic Neural Models with the GEneral NEural SImulation System.}  New York:
TELOS/Springer--Verlag.


[3] Hasselmo, M.E., Schnell, E.\ \& Barkai, E.\ (1995) Dynamics of learning and
recall at excitatory recurrent synapses and cholinergic modulation in rat
hippocampal region CA3. {\it Journal of Neuroscience} {\bf 15}(7):5249-5262.
}


%%%%%%%%%%%%%%%%%%%%%%%%%%%%%%%%%%%%%%%%%%%%%%%%%%%%%%%%%%%%
\section*{Checklist}


%%% BEGIN INSTRUCTIONS %%%
The checklist follows the references.  Please
read the checklist guidelines carefully for information on how to answer these
questions.  For each question, change the default \answerTODO{} to \answerYes{},
\answerNo{}, or \answerNA{}.  You are strongly encouraged to include a {\bf
justification to your answer}, either by referencing the appropriate section of
your paper or providing a brief inline description.  For example:
\begin{itemize}
  \item Did you include the license to the code and datasets? \answerYes{See Section~\ref{gen_inst}.}
  \item Did you include the license to the code and datasets? \answerNo{The code and the data are proprietary.}
  \item Did you include the license to the code and datasets? \answerNA{}
\end{itemize}
Please do not modify the questions and only use the provided macros for your
answers.  Note that the Checklist section does not count towards the page
limit.  In your paper, please delete this instructions block and only keep the
Checklist section heading above along with the questions/answers below.
%%% END INSTRUCTIONS %%%




\begin{enumerate}
  \item Did you include the code, data, and instructions needed to reproduce the main experimental results (either in the supplemental material or as a URL)?
    \answerTODO{}
  \item Did you specify all the training details (e.g., data splits, hyperparameters, how they were chosen)?
    \answerTODO{}
        \item Did you report error bars (e.g., with respect to the random seed after running experiments multiple times)?
    \answerTODO{}
        \item Did you include the total amount of compute and the type of resources used (e.g., type of GPUs, internal cluster, or cloud provider)?
    \answerTODO{}

\end{enumerate}


%%%%%%%%%%%%%%%%%%%%%%%%%%%%%%%%%%%%%%%%%%%%%%%%%%%%%%%%%%%%


\appendix


\section{Appendix}


Optionally include extra information (complete proofs, additional experiments and plots) in the appendix.
This section will often be part of the supplemental material.


\end{document}